\documentclass[12pt]{article}
\usepackage[
  top=2cm,
  bottom=2cm,
  left=2.5cm,
  right=2.5cm,
  headheight=17pt, % as per the warning by fancyhdr
  includehead,includefoot,
  heightrounded, % to avoid spurious underfull messages
]{geometry} % OK
\usepackage[utf8]{inputenc}
\usepackage[T1]{fontenc}
\usepackage[french]{babel} % OK
\usepackage{helvet}
\usepackage{changepage}
\usepackage{amsmath, amssymb, amsfonts, amsthm} % OK
\usepackage{mathtools} % OK
\usepackage{enumitem}
\usepackage[utf8]{inputenc}
\usepackage{tikz}
\usepackage{enumitem}
\usepackage{stmaryrd} % OK
\usepackage{fancyhdr} % OK
\usepackage{environ}
\usepackage{titletoc}
% \usepackage{hyperref} % OK
\usepackage[hidelinks]{hyperref} % OK
%% Lettres Ensembles
    \def\bA{\mathbb{A}}
    \def\bB{\mathbb{B}}
    \def\bC{\mathbb{C}}
    \def\bD{\mathbb{D}}
    \def\bE{\mathbb{E}}
    \def\bF{\mathbb{F}}
    \def\bG{\mathbb{G}}
    \def\bH{\mathbb{H}}
    \def\bI{\mathbb{I}}
    \def\bJ{\mathbb{J}}
    \def\bK{\mathbb{K}}
    \def\bL{\mathbb{L}}
    \def\bM{\mathbb{M}}
    \def\bN{\mathbb{N}}
    \def\bO{\mathbb{O}}
    \def\bP{\mathbb{P}}
    \def\bQ{\mathbb{Q}}
    \def\bR{\mathbb{R}}
    \def\bS{\mathbb{S}}
    \def\bT{\mathbb{T}}
    \def\bU{\mathbb{U}}
    \def\bV{\mathbb{V}}
    \def\bW{\mathbb{W}}
    \def\bX{\mathbb{X}}
    \def\bY{\mathbb{Y}}
    \def\bZ{\mathbb{Z}}
%%

%% Lettres cal
    \def\cA{\mathcal{A}}
    \def\cB{\mathcal{B}}
    \def\cC{\mathcal{C}}
    \def\cD{\mathcal{D}}
    \def\cE{\mathcal{E}}
    \def\cF{\mathcal{F}}
    \def\cG{\mathcal{G}}
    \def\cH{\mathcal{H}}
    \def\cI{\mathcal{I}}
    \def\cJ{\mathcal{J}}
    \def\cK{\mathcal{K}}
    \def\cL{\mathcal{L}}
    \def\cM{\mathcal{M}}
    \def\cN{\mathcal{N}}
    \def\cO{\mathcal{O}}
    \def\cP{\mathcal{P}}
    \def\cQ{\mathcal{Q}}
    \def\cR{\mathcal{R}}
    \def\cS{\mathcal{S}}
    \def\cT{\mathcal{T}}
    \def\cU{\mathcal{U}}
    \def\cV{\mathcal{V}}
    \def\cW{\mathcal{W}}
    \def\cX{\mathcal{X}}
    \def\cY{\mathcal{Y}}
    \def\cZ{\mathcal{Z}}
%%

%% Lettres gras
    \def\fA{\mathbf{A}}
    \def\fB{\mathbf{B}}
    \def\fC{\mathbf{C}}
    \def\fD{\mathbf{D}}
    \def\fE{\mathbf{E}}
    \def\fF{\mathbf{F}}
    \def\fG{\mathbf{G}}
    \def\fH{\mathbf{H}}
    \def\fI{\mathbf{I}}
    \def\fJ{\mathbf{J}}
    \def\fK{\mathbf{K}}
    \def\fL{\mathbf{L}}
    \def\fM{\mathbf{M}}
    \def\fN{\mathbf{N}}
    \def\fO{\mathbf{O}}
    \def\fP{\mathbf{P}}
    \def\fQ{\mathbf{Q}}
    \def\fR{\mathbf{R}}
    \def\fS{\mathbf{S}}
    \def\fT{\mathbf{T}}
    \def\fU{\mathbf{U}}
    \def\fV{\mathbf{V}}
    \def\fW{\mathbf{W}}
    \def\fX{\mathbf{X}}
    \def\fY{\mathbf{Y}}
    \def\fZ{\mathbf{Z}}
%%


%% Rajout des limits
    \let\oldsum\sum
    \renewcommand{\sum}[0]{\oldsum\limits}
    \let\oldprod\prod
    \renewcommand{\prod}[0]{\oldprod\limits}
    \let\oldbigcap\bigcap
    \renewcommand{\bigcap}[0]{\oldbigcap\limits}
    \let\oldbigcup\bigcup
    \renewcommand{\bigcup}[0]{\oldbigcup\limits}
    \let\oldlim\lim
    \renewcommand{\lim}[0]{\oldlim\limits}
    \let\oldsup\sup
    \renewcommand{\sup}[0]{\oldsup\limits}
    \let\oldinf\inf
    \renewcommand{\inf}[0]{\oldinf\limits}
%%


%% Important
    % \let\important\underline
    \let\important\textbf
%%

%% Divers
    \def\cad{c'est-à-dire}
    \newcommand{\todo}[1]{\textbf{TODO\@: #1}}
    \newcommand{\precise}[2]{\underset{#1}{\underbrace{#2}}}
    %% Ensemble d'entiers
        \renewenvironment{displaymath}{$$}{$$}
        \def\[{\llbracket}
        \def\]{\rrbracket}
    %%
%%
\theoremstyle{plain}
\newtheorem{thm}{Théorème}[section]
\newtheorem{thms}[thm]{Théorèmes}
\newtheorem{cor}[thm]{Corollaire}
\newtheorem{cors}[thm]{Corollaires}
\newtheorem{lem}[thm]{Lemme}
\newtheorem{lems}[thm]{Lemmes}
\newtheorem{propriete}[thm]{Propriété}
\newtheorem{proprietes}[thm]{Propriétés}
\newtheorem{proposition}[thm]{Proposition}
\newtheorem{propositions}[thm]{Propositions}

\theoremstyle{definition}
\newtheorem{definition}[thm]{Définition}
\newtheorem{definitions}[thm]{Définitions}
\newtheorem{terminologie}[thm]{Terminologie}
\newtheorem{terminologies}[thm]{Terminologies}
\newtheorem{notation}[thm]{Notation}
\newtheorem{notations}[thm]{Notations}


\theoremstyle{remark}
\newtheorem{exemple}[thm]{Exemple}
\newtheorem{exemples}[thm]{Exemples}
\newtheorem{exo}{Exercice}[section]
\newtheorem{exos}{Exercices}[section]
\newtheorem{remarque}[thm]{Remarque}
\newtheorem{remarques}[thm]{Remarques}
\newtheorem{rappel}[thm]{Rappel}
\newtheorem{rappels}[thm]{Rappels}

\setlist[itemize]{label=-}
\let\line\textit
\setlength{\parindent}{1em}
\setlength{\parskip}{.5em}
\numberwithin{equation}{subsection}
\pagestyle{fancy}
\fancyhead[C]{}
\fancyhead[L]{\leftmark}
\fancyhead[R]{\rightmark}
\fancyfoot[L,C]{}
\fancyfoot[C]{\thepage}
\renewcommand{\headrulewidth}{1pt}
\renewcommand{\footrulewidth}{1pt}
\setlength{\headheight}{30pt}
\addtolength{\topmargin}{-13pt}
\usepackage[french]{babel}
\author{Félix SASSUS BOURDA}
\title{
    Rapport de stage
}

\begin{document}
\large
\maketitle
\newpage
% \tableofcontents
% \newpage

\section{Introduction}
\todo{Présenter types intersection + le papier formalisé}

\todo{Présenter pertinence formalisation}

\newpage

\section{État de l'art}
\todo{Présenter indice de Bruijn}



\todo{Présenter locally nameless}



\todo{
  \href
    {https://bentnib.org/posts/2020-08-13-non-idempotent-intersection-types.html}
    {Article à présenter}
}

\newpage

\section{Problème traité et solution proposée}
\subsection{Choix d'implémentations}

\subsubsection{Termes}

2 approches:
\paragraph{Représentations avec des indices de Bruijn}
Avantages:
\begin{itemize}
  \item Simples à représenter
  \item Tout terme est bien formé, pas de distinction à faire entre fv et bv
\end{itemize}

Incovénients:
\begin{itemize}
  \item Parfois, distinction entre fv et bv nécessaire (par exemple pour savoir si elle est dans un contexte ou pas)
  \item Tout terme est bien formé, pas de distinction à faire entre fv et bv
\end{itemize}
\paragraph{Représentations en Locally Nameless}

\subsubsection{Types}

\subsubsection{Contextes}



\todo{
  Expliquer différents choix d'implémentations
}

\todo{
  Expliquer traduction des définitions et énoncés: 
  \begin{itemize}
    \item Comment sont formalisés les concepts
    \item justifier que ce sont de bonnes formalisations
    \item éventuellement montrer quelques fragments de code (énoncés)
  \end{itemize}
}
\todo{
  Faire un plan des étapes réalisés, des étapes restantes, évaluer les endroits qui nécessiteront + de travail, qui représentent + de risque
}

\newpage

\section{Conclusion et perspectives}
\todo{Conclusion: Résumé rapide des 2 parties précédentes}
\todo{
  Perspectives:
  \begin{itemize}
    \item Compléter la preuves:
      \begin{itemize}
        \item Lemmes techniques auquels on a confiance
        \item pts critiques de la preuves
        \item ajouter un nouveau résultats ?
      \end{itemize}
      Commenter sur la difficulté de chacun (simple si on prend le temps de le faire, nécessite de nouvelles idées/nouveaux résultats intermédiaires)
    \item Revenir sur le TER (commenter l'approche LN par exemple)
  \end{itemize}
}

\newpage

\end{document}

